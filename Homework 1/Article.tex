\documentclass[serif]{article-Bram}

\title{AMCS 394E Homework 1}
\author{{Bram Lagerweij\thanks{Corresponding student: KAUST COHMAS department Building 4, room 2610, \href{mailto:abraham.lagerweij@kaust.edu.sa}{abraham.lagerweij@kaust.edu.sa}}}}

\begin{document}
\begin{abstract}
	Homework regarding the first week. The goal is  to  work with basic numerical approximation of PDE's' and functions.
\end{abstract}

\section{Using the method of lines}
Consider the one-dimensional advection diffusion equation:
\begin{equation}\label{eq:PDE}
	\pdv{u}{t} + \pdv{u}{x} + μ\pdv[2]{u}{x} = 0 \qquad ∀x ∈ Ω = [0, 1] \qand t>0
\end{equation}
where $\mu>0$ is a coefficient.  Consider periodic boundary conditions and the following initial condition:
\begin{equation}
	u(x,0) = \sin(2πx)
\end{equation}
What do we expect the exact solution to do? Due to the advective part, the initial condition travels at constant speed to the right. At the same time, due to the diffusive term, the initial condition is dissipated at a rate that depends on $μ$.

Consider the following discretization. Use second-order central finite differences to approximate $u_x$ and $u_{xx}$. Use forward and backward Euler to obtain full discretization (write down the schemes).
Consider a fixed mesh with $∆x=10^{−2}$ .

\subsection*{Exersize 1}
Consider a final time of $t=1$ and $μ=0.01$. For each full discretization proceed as follows:
\begin{itemize}[nosep]
	\item Experiment using the following time step sizes: $∆t = 10^{−4}$, $10^{−3}$, $10^{−2}$ and $10^{−1}$ .
	\item How do the explicit and implicit methods behave for these time steps?
\end{itemize}


\subsection*{Exersize 2}
Consider $μ=0$ and solve \cref{eq:PDE} using the explicit and the implicit methods. Use $∆t = 10^{−4}$ and solve the problem for the following final times: $t=1$, 5, 10, 15 and 20. Comment on the behaviour of each full discretization as the final time increases.

\section{Approximation of functions}
Consider the function
\begin{equation}\label{eq:exact}
	f(x) = \sin[4](2πx) \qquad ∀x ∈ Ω = [0, 1]
\end{equation}
for which we have to find multiple global and local approximations. Let $f_h (x)$ be such an approximation for a given grid. We consider the following errors:
\begin{equation*}
	E_1 := \int_Ω \abs{f(x) - f_h(x)} \dd{x} \qand E_2 := \int_Ω \qty(f(x) - f_h(x))^2 \dd{x}
\end{equation*}

\subsection*{Exersize 3: Global approximations}

Consider the following approximations all with $N$ terms:
\begin{enumerate}[a), nosep]
	\item the Taylor series around $x=0.5$,
	\item the Fourier series,
	\item a global polynomial interpolation given by
	\begin{equation*}
		f_h(x)=a_0 + a_1x + a_2 x^2 + \dots
	\end{equation*}
	with $f_h(x_i)=f(x_i)$ for an evenly spaced set of $N$ points on the closed interval.
\end{enumerate}
Consider different levels of refinement, $N=4,5,6,\dots,10$ and for each approximation report both $E_1$ and $E_2$

\subsection*{Exercize 4: Local approximations}
Split the domain $Ω$ into $N$ cells. For each cell $K$, compute linear and quadratic approximations $f_K(x)$ with $f_K(x_i) = f(x_i)$ where $x_i$ are evenly spaced grid points (including the boundaries of the cell) within cell $K$. Compute and report the errors $E_1$ and $E_2$ for different number of cells; e.g., $N = 4, 5, 6, . . . , 10$.



\end{document}